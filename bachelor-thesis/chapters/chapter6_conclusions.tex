\chapter{Conclusions}
%\chapter*{Conclusions}
\label{conclusions}
Significant improvements in the safety and flexibility of cryptocurrency transactions can be achieved using Vault, a powerful solution for controlling the security of digital assets and cryptocurrencies on the Ethereum chain.

By requiring many owners to approve transactions, multisig, or multiple signature, increases security by lowering the risks involved in managing a single access point. Even in the event that a private key is compromised, this approach restricts the likelihood of financial theft and prohibits unauthorized access.

The Atlas feature of the application is among its most inventive features. This feature adds an extra degree of protection and verification before transactions are finalized, much like the two-step authentication (2FA) systems seen in web-based apps. 

Because of its great versatility, the wallet may be configured in a variety of ways, from a single owner and approval level to simulating a 2FA wallet with Atlas capability. Users can adjust the security level to suit their own needs thanks to this. Additionally, money transfers require the 2FA code, even in the event that the single private key is compromised.

There are two primary areas that could be further addressed in terms of improvements. Since the contract already accepts ERC20 tokens, the first step would be to incorporate support for them directly into the application. By doing this, the wallet's functionality would be increased to support a larger variety of cryptocurrency assets. The second would be the development of a "Transaction Builder" page, such to what Safe provides \cite{transactionBuilder}, that would enable users to construct intricate transactions in a clear and simple manner, hence improving accessibility and transaction management efficiency.

In summary, Vault stands out for its successful synthesis of cutting-edge security, originality, and adaptability. It also has a great deal of room for future growth to better serve user demands.
