\chapter{Technologies}

\section{Solidity}
\label{sec:ch3sec1}
Solidity is a programming language that was created specifically for Ethereum smart contract creation \cite{sdocs} .The primary features are:

\begin{itemize}
	\item OOP: Solidity is an object-oriented programming language, which makes code organization and reuse simple. It is built on the idea of classes and objects.
	\item JavaScript-Like: Because of its core structure, which is similar to that of JavaScript, web professionals may easily learn Solidity. Its quirks have been tailored especially for blockchain, nevertheless.
	\item Solidity utilises static typing, which necessitates the specification of the data type of variables in advance. By doing this, compile-time errors are caught and contracts are enforced..
\end{itemize}

However given that smart contracts are deployed once and cannot be altered after you need to be very carefull about potential security risks that can lead to stolen funds. Some of the most common vulnerabilities are: Reentrancy Attacks, Integer Overflow/Underflow, Uninitialized Storage Pointers, Denial of Service (DoS) Attacks \cite{solidityVulnerabilities}

\section{Hardhat}
\label{sec:ch3sec2}
Hardhat is an open-source testing and development framework for Ethereum. The open-source framework for Ethereum development and testing \cite{hardhat}. It improves the process for creating and testing smart contracts. Hardhat is flexible; it supports numerous Solidity compilers and allows for different network configurations. Additionally, it may be expanded upon by combining it with additional blockchain ecosystem tools and frameworks. The comprehensive range of testing features contributes to the security and reliability of smart contracts. Hardhat strongly speeds up development, enhances the quality of the code, and lowers the possibility of mistakes in contracts that are well-written.

\section{Nextjs}
\label{sec:ch3sec3}
Next.js is an open-source React framework that extends the core React capabilities to create high-performance and SEO-friendly web applications. NextJs allows components to be rendered on the server, not only does it help with speed because the server can prerender the response but it can also help with webcrawlers to scan your site. The best features of NextJs include:
\begin{itemize}
	\item Image optimizations: "Automatically serve correctly sized images for each device, using modern image formats like WebP and AVIF." \cite{nextImage}
	\item Dynamic HTML Steaming
	\item Client And Server Components
	\item Server Actions: "Server Actions are asynchronous functions that are executed on the server. They can be used in Server and Client Components to handle form submissions and data mutations in Next.js applications." \cite{nextServerActions}
	\item Route Handlers: "allow you to create custom request handlers for a given route using the Web Request and Response APIs." \cite{nextRouteHandlers}
\end{itemize}

\section{Typescript}
\label{sec:ch3sec4}
"TypeScript adds additional syntax to JavaScript to support a tighter integration with your editor. Catch errors early in your editor"\cite{typescript}. Static types contribute to the reliability and maintainability of your code by reducing compilation errors. Classes make code more readable and scalable by enabling it to be divided into reusable modules. Compiling TypeScript into plain JavaScript makes it interoperable with every runtime and browser.
Using TypeScript has several advantages:
\begin{itemize}
\item Boost code quality: Static types make code more robust by assisting in the detection and prevention of compilation issues.
\item Boosts scalability: Code may be arranged into reusable modules by using classes and modules, which makes it simpler to create bigger projects.
\item Boost teamwork: Integrated documentation and static types can help enhance teamwork and communication.
\end{itemize}

\section{Tailwind}
\label{sec:ch3sec5}
Tailwind is defined as "a utility-first CSS framework packed with classes like flex, pt-4, text-center and rotate-90 that can be composed to build any design, directly in your markup" \cite{tailwind}.

Some of the Tailwind CSS's primary features are:
\begin{itemize}
\item Utility Classes: An extensive library of pre-made classes that style HTML components without requiring the creation of unique CSS code.
\item Components that can be customized: Enables the development of reusable parts with unique styles.
\item Expandable: Plugins can be added to expand the capabilities.
\end{itemize}
The Benefits of tailwind is that it drastically accelerates development, less CSS code writing is needed. There is no need in thinking what to name your css classes because they are pre defined. It minimizes the CSS file size, only the css needed is kept, the rest is "purged"

\section{RainbowKit}
\label{sec:ch3sec6}
RainbowKit is an open-source SDK that makes it it easier to incorporate Web3 wallets into your web apps. It gives consumers a range of alternatives by supporting multiple blockchains, including nearly every EVM compatible chain. RainbowKit's user-friendly interface lowers friction while facilitating a simple and seamless connection experience. Because of its customization capabilities, you can incorporate Web3 features unique to your project and alter the wallet's appearance. Rainbowkit also allows to easily change your chain whenever you like \cite{rainbowkit}

\section{Infura}
\label{sec:ch3sec7}
Infura is a platform that gives you access to blockchain nodes, drastically simplifying the process of connecting and interacting with blockchains such as Ethereum, IPFS, and Polygon, etc. By using Infura, you can save time and money by avoiding the hassle of configuring your own nodes. With the help of this platform's high scalability, security features, and quick access to blockchain data, you may create decentralized apps (dApps), incorporate blockchain technology into already-existing apps, and test concepts quickly without needing to design complicated infrastructure. \cite{infura}

\section{Etherscan}
\label{sec:ch3sec8}
The Etherscan API allows developers to access and query data from the Ethereum blockchain. It offers a wide range of functions that allow:
\begin{itemize}
	\item Get details about particular blocks, transactions, and accounts: You may find out information about balances, contract codes, and transaction histories for particular blocks, transactions, and accounts.
	\item Sending Transactions: You can call smart contract operations and transmit ETH transfers to the Ethereum network.
	\item Subscribe to events: You can get alerts in real time when new blocks or ETH transfers occur on the blockchain.
	\item Historical data query: The Ethereum blockchain provides historical information on blocks, transactions, and account statuses.\cite{etherscan}
\end{itemize}

\section{Viem}
\label{sec:ch3sec9}
Viem is a library that makes it easier and more natural for developers to work with the Ethereum network. Viem offers an abstraction over the implementation of the rpc methods used by EVM.\cite{viem}
Important aspects of Viem:
\begin{itemize}
\item Higher-level abstractions: Viem makes it simpler to work with notions like accounts, transactions, and smart contracts in your code by providing higher-level abstractions for them.
\item Defined Data Types: Viem increases the safety and readability of your code by predefining the data types that ethereum will use.
\item Multi-Chain Support: Polygon and Avalanche are two of the many Ethereum-compatible blockchains that Viem supports.
\item Simple integration: Viem interfaces with several well-known JavaScript frameworks and libraries with ease.
\end{itemize}

\section{Wagmi}
\label{sec:ch3sec10}
Wagmi is a library that greatly simplifies web3 development when utilizing React. To make communicating with the blockchain easier, Wagmi provides over 20 react hooks. These hooks can be used to get transactions, listen to events, or even communicate with other addresses or smart contracts. Additionally, Wagmi is the authorized connector for EIP-6963, walletconnect, and metamask. Tanstack useQuery enables Wagmi to assist with caching the get requests and avoiding deduplication. \cite{wagmi}

\section{useQuery}
\label{sec:ch3sec11}
useQuery is a library for the react framework that helps you with fetching data from external servers and doing mutations. It helps you with a lot of common task when dealing with this kind of use cases. This are some of its features
\begin{itemize}
	\item Prevents deduplication thanks to query keys
	\item Data Caching
	\item Error Handling
	\item Loading and Refetching handling
	\item Revalidating data after mutations
	\item Time based revalidation
\end{itemize}

\section{HTML}
\label{sec:ch3sec12}
HTML (HyperText Markup Language) is the standard language used to create and structure web pages. It was developed by Tim Berners-Lee in the early 1990s and has since become central to web development. HTML is used to define the content of web pages, including text, images, links, forms, and other multimedia elements.

The basic structure of an HTML document includes several main elements. The first of these is DOCTYPE html, which declares the document type and version of HTML used. This is usually the first line in an HTML file. The next important element is html, which is the root element and frames the entire content of the document.

The head element contains meta-information about the document, such as the page title, links to CSS files, JavaScript scripts, and other relevant data. The title of the web page is defined by the title element and appears in the title bar of the browser. The main content of the web page is placed in the body element, which can include text, images, links, and other elements.

To structure content, HTML provides various elements. Headings are defined using h1 through h6 elements, each representing a different level of importance. Paragraphs are defined by the p element. Links are created using the a element. Lists are structured using the ul elements for unordered lists and the ol elements for ordered lists, each element in the list being marked with li.

Images are embedded in the page using the img element, specifying the src attribute for the image source and the alt attribute for the alt text. Tables are created with the table element, using tr for rows, th for header cells, and td for data cells. Forms are built using the form element along with input, text, select and button for various types of inputs and controls.

HTML elements are structured hierarchically. Block elements, such as div and section, can contain other block elements or inline elements, such as span and a. This structure allows the logical and visual organization of web page content, ensuring a coherent and accessible user experience.

\section{Javascript}
\label{sec:ch3sec13}
JavaScript is a versatile and dynamic programming language used mainly for developing interactive web pages. It was created by Brendan Eich in 1995 during his time at Netscape Communications Corporation, and quickly became one of the most essential tools for web developers.

JavaScript allows developers to add interactive and dynamic behaviors to web pages. It can be used to manipulate the Document Object Model (DOM), which represents the HTML structure of a web page, allowing the content and style of the page to be changed in real-time without requiring a page reload.

One of the key aspects of JavaScript is its ability to run directly in the user's browser. This means that JavaScript can react to user actions such as mouse clicks, text inputs and other events, improving the user experience with fast and interactive responses.

JavaScript is often used in conjunction with HTML and CSS. While HTML defines the structure of a web page's content and CSS manages its style and appearance, JavaScript adds functionality and behavior. For example, a button on an HTML page can be made to trigger an animation or send data to a server when pressed, thanks to JavaScript.

Another major advantage of JavaScript is its extensive support for libraries and frameworks that simplify web development. Popular libraries like jQuery, and powerful frameworks like React, Angular and Vue.js, provide predefined solutions and frameworks to develop complex web applications faster and more efficiently.

\section{CSS}
\label{sec:ch3sec14}
CSS (Cascading Style Sheets) is a styling language used to describe the appearance and format of a document written in a markup language such as HTML. Developed by the W3C (World Wide Web Consortium), CSS allows web developers to control the visual design and presentation of web pages by separating content from style.

One of the main advantages of CSS is its ability to apply consistent and reusable styles to multiple web pages. By using external CSS files, developers can define styles once and apply them throughout the website, simplifying the process of maintaining and updating the design.

CSS works based on rules that consist of selectors and declarations. Selectors identify the HTML elements that will be styled, while declarations specify the style properties that will be applied to those elements. Declarations consist of property-value pairs, such as color: blue or font-size: 16px.

CSS provides a wide range of properties that allow you to customize the appearance of HTML elements. These include properties for controlling colors, fonts, margins, spacing, sizes, positioning, and more. By combining these properties, developers can create complex and attractive layouts.

One of the key features of CSS is "cascading", which determines how styles are applied when there are multiple conflicting rules for the same element. Cascadea uses a priority system based on specificity, order of declarations, and source of styles (inline styles, external files, or browser default styles).
